\documentclass[dvipdfmx,11pt]{beamer}
\usepackage{float}
\usepackage{pxjahyper}
\usepackage{minijs,amsmath,comment}
\renewcommand{\kanjifamilydefault}{\gtdefault}
\usetheme{Antibes}
\setbeamertemplate{navigation symbols}{}
\setbeamertemplate{footline}[page number]

\title{ゼミについて}
\author{明治大学数学科3年 宮部研究室 奥城健太郎}
\institute{宮部研究室3年生歓迎会}
\date{2017年9月23日}

\begin{document}

\begin{frame}
\frametitle{}
\titlepage
\end{frame}

\begin{frame}
\frametitle{目次}
\begin{itemize}
 \item 今後のゼミの予定
 \item ゼミの種類について
 \item ゼミの準備について
 \item まとめ
\end{itemize}
\end{frame}

\begin{frame}
\frametitle{今後のゼミの予定}
3年秋 ゼミナールB
4年春 卒業研究1
4年秋 卒業研究2
\end{frame}

\begin{frame}
ゼミナールBは卒業研究のための準備期間
\end{frame}

\begin{frame}
\frametitle{ゼミの準備について}
秘書問題とは,
\begin{enumerate}
 \item<2-> 秘書を一人雇いたい.
 \item<3-> n人が応募してきている. nは既知とする.
 \item<4-> 応募者には重複無く順位が付けられる.
 \item<5-> 無作為に面接を行う.
 \item<6-> 毎回の面接の後,採用するかどうかをその場で決定する.
 \item<7-> その応募者を採用するかどうかは,それまでの相対的順位のみによって決定する.
 \item<8-> 不採用にしたら後から採用することはできない.
 \item<9-> 最も良い応募者を選ぶ確率を高くするにはどうすれば良いか.
\end{enumerate}
\end{frame}

\begin{frame}
\begin{block}{戦略}
最初のr人は不採用とする.

その後の面接者の中で,それまでの応募者の中で最も良ければ採用する.
\end{block}

\begin{quote}
最善の応募者を選択できる確率を求めたい.

求める確率が最も高くなるようなrは?

また,そのときの確率は?
\end{quote}
\end{frame}

\begin{frame}
最善の応募者がi番目にいる確率は$ \frac{1}{n} $. $ (r+1 \leq i \leq n) $

$i-1$番目までの人の中で最高の順位の人が

r番目までにいる確率は$ \frac{r}{i-1} $

よって,求める確率を$p(r)$とすると,
\[ p(r) = \sum_{i=r+1}^n \frac{1}{n} \frac{r}{i-1} \]
である.

\end{frame}

\begin{frame}
$ x= \frac{r}{n} $とすると,
\begin{align*}
p(r) &= \sum_{i=r+1}^n \left( \frac{1}{n} \cdot \frac{nx}{i-1} \right)  \\
      &= \sum_{i=r+1}^n \left( \frac{x}{n} \cdot \frac{1}{\frac{i-1}{n}} \right)  \\
      &= \frac{x}{n} \left( \frac{1}{\frac{r}{n}} + \frac{1}{\frac{r+1}{n}} + \ldots + \frac{1}{\frac{n-1}{n}} \right)
\end{align*}
nが十分大きいとき,区分求積法を用いて,
\[ p(r) \to x \int_{\frac{r}{n}}^{1} \frac{dt}{t} = -x \log x \]
\end{frame}

\begin{frame}
\[ g(x) = -x \log x \]
とすると,これが最大となるようなxは,
\[ g'(x) = -\log x-1 = 0 \]
を解いて$ x = \frac{1}{e} $.

$ x = \frac{r}{n} $なので, $ r = \frac{n}{e} $

このとき,最善の応募者を選択できる確率は,
\[ g \left( \frac{1}{e} \right) = \frac{1}{e} \]
\end{frame}

\begin{frame}
最も良い応募者を選択するには,
\begin{itemize}
 \item 最初の37%の人をすべて不採用にする.
 \item その後の面接者の中で,それまでの応募者の中で最も良ければ採用する.
 \item この戦略が成功する確率は37%である.
\end{itemize}
\end{frame}

\begin{frame}
\frametitle{今後の予定}
\begin{itemize}
 \item 統計の基礎的な部分を学習する.
 \item 統計に関する卒業研究を行う.
\end{itemize}
\end{frame}

\begin{frame}
\frametitle{参考文献}
宮部研究室ゼミナールB講義録
\end{frame}

\end{document}